% !TEX encoding = UTF-8
% !TEX TS-program = pdflatex
% !TEX root = ../tesi.tex
% !TeX spellcheck = it_IT

%**************************************************************
% Sommario
%**************************************************************
\cleardoublepage
\phantomsection
\pdfbookmark{Sommario}{Sommario}
\begingroup
\let\clearpage\relax
\let\cleardoublepage\relax
\let\cleardoublepage\relax

\chapter*{Sommario}
Il presente documento presenta la relazione finale redatta in seguito al completamento dell'esperienza di stage di due mesi effettuata presso l'azienda RiskApp.
\'E così composto:
\begin{enumerate}
    \item descrizione del contesto aziendale in cui mi sono inserito;
    \item descrizione degli obiettivi di stage;
    \item note di teoria sugli argomenti che ho poi affrontato dal punto di vista produttivo;
    \item descrizione in dettaglio della progettazione del software prodotto;
    \item valutazione retrospettiva dell'esperienza;
    \item appendice dove sono presentati i risultati sperimentali prodotti dal software.
\end{enumerate}

\section*{Note tecniche}
Riguardo la stesura del testo sono state adottate le seguenti convenzioni tipografiche:
\begin{itemize}
    \item gli acronimi, le abbreviazioni e i termini ambigui o di uso non comune menzionati vengono definiti nel glossario, situato alla fine del documento;
    \item per la prima occorrenza dei termini riportati nel glossario viene utilizzato il colore azzurro;
    \item i termini in lingua straniera o che fanno parte del gergo tecnico sono evidenziati con il carattere \emph{corsivo};
    \item per termini che si riferiscono a prodotti software si utilizzerà invece il colore \grayname{grigio}.
\end{itemize}
%\begin{itemize}
%	\item Cos'è il documento
%	\item Lo scopo dello stage in generale
%	\item Specifica occupazione
%	\item cosa è stato creato
%	\item breve riassunto delle tecnologie usate -> riferimento all'Appendice
%\end{itemize}


%\vfill
%
%\selectlanguage{english}
%\pdfbookmark{Abstract}{Abstract}
%\chapter*{Abstract}
%
%\selectlanguage{italian}

\endgroup			

\vfill

